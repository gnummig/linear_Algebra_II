\documentclass[a4paper,10pt]{article}
\usepackage[utf8]{inputenc}
\usepackage{amsmath}
\usepackage{amssymb}   % for number sets
\usepackage{xfrac} % for nicefrac


\newcommand{\RN}[1]{%
  \textup{\uppercase\expandafter{\romannumeral#1}}%
}
%opening
\title{}
\author{}

\begin{document}

\maketitle



\section{determine the solution set of the following systems of equations}
\subsection*{a1}
solve for real numbers:
\[
\begin{bmatrix}
2 & 3  & 1  & 2 \\
4 & 3  & 1  & 1  \\
5 & 11 & 3  & 2  \\
2 & 5  & 1  & 1  \\
1 & -7 & -1 & 2  
\end{bmatrix}
\times
\begin{bmatrix}
 x_1\\x_2\\x_3\\x_4
\end{bmatrix}
=
\begin{bmatrix}
 4 \\ 5\\2 \\ 1 \\ 7
\end{bmatrix}
\]
solution:\\
since there are 5 equations for 4 variables, we do not expect to necessarily find a solution, still we can proceed solving the system of the first four equations and checking whether the solution is valid for the last equation.\\
\[
\begin{bmatrix}
2 & 3  & 1 & 2 & \vline & 4\\
4 & 3  & 1 & 1 & \vline & 5\\
5 & 11 & 3 & 2 & \vline & 2\\
2 & 5  & 1 & 1 & \vline & 1\\
\end{bmatrix}
\quad
\begin{matrix}
 \RN{1}\\ \RN{2}\\ \RN{3}\\ \RN{4}
\end{matrix}
\]
we can reduce the problem to solving for $x_2,x_3,and x_4$
\[
\begin{bmatrix}
 3 & 1 & 3 & \vline & 3\\
 7 & 1 & -6& \vline & -16\\
 2 & 0 & -1 & \vline & -3
\end{bmatrix}
\quad
\begin{matrix}
 \RN{5} = 2\times \RN{1}-\RN{2} \\ \RN{6}=2 \times \RN{3}-5 \times \RN{1} \\\RN{7}= \RN{4}-\RN{1}
\end{matrix}
\]
which is the same as solving:\\
\[
\begin{bmatrix}
 4 & 39 & \vline & 69\\
 2 & 9 & \vline & 15
\end{bmatrix}
\quad
\begin{matrix}
 \RN{8}=7\times \RN{5}-3\times \RN{6} \\\RN{9}= 2\times \RN{5} - 3\times \RN{7}
\end{matrix}
\]
from which it is simple to find
\[
x_4 = \frac{39}{21}= \frac{13}{7}
\]
therefore using $\RN{9}$
\[
x_3= \frac{105-117}{14}= -\frac{13}{14}
\]
using $\RN{7}$ we find:
\[
x_2 = \frac{13-21}{14}= -\frac{4}{7}
\]
and inserting in $\RN{1}$ we find:

\[
x_1= \frac{12}{14} +\frac{13}{28} -\frac{26}{14}+2= \frac{24+13-52+54}{28}=\frac{39}{28}
\]
now testing this in the last equation of the exercise, we get:
\[
\frac{39}{28} + 4 +\frac{13}{14} +\frac{26}{7}=\frac{39+102+26+104}{28}= \frac{271}{28} \neq 7
\]
Hence the solution of the reduced system of equations does not solve the entire system and the solution set is 
\[
S = \emptyset
\]
\subsection*{a2}
\[
\begin{bmatrix}
  9 & -3 & 5 & 6 \\
  6 & -2 & 3 & 1\\
  3 & -1 & 3 & 14
\end{bmatrix}
\times
\begin{bmatrix}
 x_1\\x_2\\x_3\\x_4
\end{bmatrix}
=
\begin{bmatrix}
 4 \\ 5\\-8
\end{bmatrix}
\]

since there are 3 constrains in four dimensions, we expect the solution set to be at least a hyperplane of dimension 1
lets set $x_4=a \in \mathbb{R}$
\[
\begin{bmatrix}
  9 & -3 & 5 & 6  &\vline  & 4 \\
  6 & -2 & 3 & 1  &\vline  & 5 \\
  3 & -1 & 3 & 14 &\vline  & -8\\
  0 &  0 & 0 & 1  &\vline  & a
\end{bmatrix}
\begin{matrix}
 \RN{1}\\ \RN{2}\\ \RN{3}\\ \RN{4}\\
\end{matrix}
\]
\[\begin{bmatrix}
  9 & -3 & 5 & 6  &\vline  & 4 \\
  0 & 0 & -4 & -36  &\vline  & 28\\
  0 & 0 & -3 & -27 &\vline  & 21\\
  0 &  0 & 0 & 1  &\vline  & a
\end{bmatrix}
\begin{matrix}
 \RN{1}\\ \RN{5}= \RN{1}-3\times \RN{3}\\ \RN{6}=\RN{2}-2\times \RN{3}\\ \RN{4}\\
\end{matrix}
\]
\[\begin{bmatrix}
  9 & -3 & 5 & 6  &\vline  & 4   \\
  0 & 0 & -4 & -36  &\vline  & 28\\
  0 & 0 & 1 & 0 &\vline  & -7-9a \\
  0 &  0 & 0 & 1  &\vline  & a
\end{bmatrix}
\begin{matrix}
 \\ \\ \RN{7}=-\frac{1}{3} \RN{6}-7\times \RN{4} \\ \\
\end{matrix}
\]
since the second row does not depend depend on $x_2$ we can choose $x_2= b \in \mathbb{R} $
\[\begin{bmatrix}
  9 & -3 & 5 & 6  &  \vline  & 4   \\
  0 & 1 & 0 & 0  &   \vline  & b\\
  0 & 0 & 1 & 0 &    \vline  & -7-9a  \\
  0 &  0 & 0 & 1  &   \vline  & a
\end{bmatrix}
\begin{matrix}
 \\ \RN{8}=-\frac{1}{4} \RN{5}-9\times \RN{4}\\ \RN{6}=\RN{2}-2\times \RN{3}\\ \RN{4}\\
\end{matrix}
\]
which leaves us with 
\[\begin{bmatrix}
  1 & 0 & 0 & 0  &  \vline  & \frac{13+13a +b}{3}  \\
  0 & 1 & 0 & 0  &   \vline  & b\\
  0 & 0 & 1 & 0 &    \vline  & -7-9a  \\
  0 &  0 & 0 & 1  &   \vline  & a
\end{bmatrix}
\]
so the solution set is of the form 
\[
S=
\begin{bmatrix}
 \sfrac{13}{3} \\ 0\\-7\\ 0\\
\end{bmatrix}
+a\begin{bmatrix}
  \sfrac{13}{3} \\0\\-9\\1
 \end{bmatrix}
+b\begin{bmatrix}
   \sfrac{1}{3} \\ 1\\0\\0
  \end{bmatrix}  
  a,b \in \mathbb{R}
\]
\subsection*{b}
solve for rational numbers:
\[
\begin{bmatrix}
 24 & 14 & 30 & 40 & 41 & \vline & 28\\
 36 & 21 & 45 & 61 & 62 & \vline & 43\\
 48 & 28 & 60 & 82 & 83 & \vline & 58\\
 60 & 35 & 75 & 99 & 102 &\vline & 69\\
\end{bmatrix}
\]
choose $x_5=a \in \mathbb{Q}$
\[
\begin{bmatrix}
 24 & 14 & 30 & 40 & 41 & \vline & 28\\
 36 & 21 & 45 & 61 & 62 & \vline & 43\\
 48 & 28 & 60 & 82 & 83 & \vline & 58\\
 60 & 35 & 75 & 99 & 102 &\vline & 69\\
 0 &  0  & 0  & 0  &   1 & \vline & a\\
\end{bmatrix}
\]
\[
\begin{bmatrix}
 24 & 14 & 30 & 40 & 0 & \vline & 28-41a\\
 36 & 21 & 45 & 61 & 0 & \vline & 43-62a\\
 48 & 28 & 60 & 82 & 0 & \vline & 58-83a\\
 60 & 35 & 75 & 99 & 0 &\vline & 69-102a\\
 0 &  0  & 0  & 0  &   1 & \vline & a\\
\end{bmatrix}
\]
\[
\begin{bmatrix}
 24 & 14 & 30 & 40 & 0 & \vline & 28-41a\\
 36 & 21 & 45 & 61 & 0 & \vline & 43-62a\\
 48 & 28 & 60 & 82 & 0 & \vline & 58-83a\\
 0 & 0 & 0 & -2 & 0 &\vline & 2(69-102a)-5(28-41a)\\
 0 &  0  & 0  & 0  &   1 & \vline & a\\
\end{bmatrix}
\]
\[
\begin{bmatrix}
 24 & 14 & 30 & 40 & 0 & \vline & 28-41a\\
 36 & 21 & 45 & 61 & 0 & \vline & 43-62a\\
 48 & 28 & 60 & 82 & 0 & \vline & 58-83a\\
 0 & 0   & 0  & 1  & 0 & \vline & 1\\
 0 &  0  & 0  & 0  & 1 & \vline & a\\
\end{bmatrix}
\]
\[
\begin{bmatrix}
 24 & 14 & 30 & 0  & 0 & \vline & 28-40-41a\\
 36 & 21 & 45 & 0  & 0 & \vline & 43-61-62a\\
 48 & 28 & 60 & 0  & 0 & \vline & 58-82-83a\\
 0 & 0   & 0  & 1  & 0 & \vline & 1\\
 0 &  0  & 0  & 0  & 1 & \vline & a\\
\end{bmatrix}
\]
\[
\begin{bmatrix}
 24 & 14 & 30 & 0  & 0 & \vline & 12-41a\\
 36 & 21 & 45 & 0  & 0 & \vline & 18-62a\\
 48 & 28 & 60 & 0  & 0 & \vline & 24-83a\\
 0 & 0   & 0  & 1  & 0 & \vline & 1\\
 0 &  0  & 0  & 0  & 1 & \vline & a\\
\end{bmatrix}
\]
\[
\begin{bmatrix}
 24 & 14 & 30 & 0  & 0 & \vline & 12-41a\\
 36 & 21 & 45 & 0  & 0 & \vline & 18-62a\\
 0  & 0  & 0  & 0  & 0 & \vline & 24-24+82a-83a\\
 0 & 0   & 0  & 1  & 0 & \vline & 1\\
 0 &  0  & 0  & 0  & 1 & \vline & a\\
\end{bmatrix}
\] 
\[
\begin{bmatrix}
 24 & 14 & 30 & 0  & 0 & \vline & 12-41a\\
 36 & 21 & 45 & 0  & 0 & \vline & 18-62a\\
 0  & 0  & 0  & 0  & 0 & \vline & a\\
 0 & 0   & 0  & 1  & 0 & \vline & 1 \\
 0 &  0  & 0  & 0  & 1 & \vline & a \\
\end{bmatrix}
\]
we follow that $a=0$ and see that the second row is a multiple of the first, hence choose $x_3=b \in \mathbb{Q} and x_2=c \in \mathbb{Q}$
\[
\begin{bmatrix}
 12 & 0 &  & 0  & 0 & \vline & 34 -15b -7c\\
 0  & 1  & 0  & 0  & 0 & \vline & c\\
 0  & 0  & 1  & 0  & 0 & \vline & b \\
 0 & 0   & 0  & 1  & 0 & \vline & 1 \\
 0 &  0  & 0  & 0  & 1 & \vline & 0 \\
\end{bmatrix}
\]

the solution set is of the form:
\[
S= 
\begin{bmatrix}
 \frac{17}{6}\\ 0\\0\\1\\0\\
\end{bmatrix}
+b
\begin{bmatrix}
 -15\\0\\1\\0\\0\\
\end{bmatrix}
+c
\begin{bmatrix}
 -7\\1\\0\\0\\0\\
\end{bmatrix}
\quad b,c \in \mathbb{Q}
\]
\subsection*{c}
solve for complex numbers
\[
\begin{bmatrix}
 -1  & i & 1+i & \vline & i\\
 3+i & 0 & -4+2i& \vline & -1-i\\
 1-i & 2 & -1+2i & \vline & 1+i\\
\end{bmatrix}
\begin{matrix}
 \RN{1}\\ \RN{2}\\ \RN{3}\\
\end{matrix}
\]
\[
\begin{bmatrix}
 3+i & 0 & -4+2i& \vline & -1-i\\
 -1  & i & 1+i & \vline & i\\
 (1-i+3+i -4)& 2+4i &( -1+2i-4+2i +4+4i) & \vline & 1+i-1-i+4i\\
\end{bmatrix}
\begin{matrix}
 \RN{2}\\ \RN{1}\\ \RN{4}=\RN{3}+\RN{2}+4\times \RN{1}\\
\end{matrix}
\]
\[
 \begin{bmatrix}
 3+i & 0 & -4+2i& \vline & -1-i\\
 -1  & i & 1+i & \vline & i\\
 0 & 2+4i & -1+8i   & \vline & 4i\\
\end{bmatrix}
\begin{matrix}

\RN{2}\\ \RN{1}\\ \RN{4}\\
\end{matrix}
\]
\[
 \begin{bmatrix}
 3+i & 0     & -4+2i & \vline & -1-i\\
 0   & i     & (3+i)(1+i) -4+2i  & \vline & ( 3+i )i -1-i\\
 0   & 2+4i  & -1+8i & \vline & 4i\\
\end{bmatrix}
\begin{matrix}
\RN{2}\\ \RN{5}= \RN{2}+(3+i)\times \RN{1} \\ \RN{4}\\
\end{matrix}
\]

\[
 \begin{bmatrix}
 3+i & 0     & -4+2i & \vline & -1-i\\
 0   & i     & -2+6i & \vline & 2i-2\\
 0   & 2+4i  & -1+8i & \vline & 4i\\
\end{bmatrix}
\]

\[
 \begin{bmatrix}
 3+i & 0     & -4+2i & \vline & -1-i\\
 0   & i     & -2+6i & \vline & 2i-2\\
 0   & 0     & (-1+8i +(4-2i)(-2+6i)) & \vline & 4i+(4-2i)(2i-2)\\
\end{bmatrix}
\begin{matrix}
\RN{2}\\ \RN{5} \\ \RN{6}=\RN{4}+(4-2i)\times \RN{5}\\
\end{matrix}
\]

\[
 \begin{bmatrix}
 3+i & 0     & -4+2i & \vline & -1-i\\
 0   & i     & -2+6i & \vline & 2i-2\\
 0   & 0     & 3+36i & \vline & -4+16i\\
\end{bmatrix}
\]
therefore we find $x_3$ as
\[
x_3= \frac{-4+16i}{3+36i}=\frac{(-4+16i)(3-36i)}{(3+36i)(3-36i)}=\frac{+192i+564}{3^2+6^4}
\]

\[
 \begin{bmatrix}
 3+i & 0     &     & \vline & -1-i- \frac{(+192i+564)(-4+2i)}{3^2+6^4}\\
 0   & i     & 0   & \vline & 2i-2-\frac{(+192i+564)(-2+6i)}{3^2+6^4}\\
 0   & 0     & 1   & \vline & \frac{+192i+564}{3^2+6^4}\\
\end{bmatrix}
\]

\[
 \begin{bmatrix}
 3+i & 0     & 0   & \vline & -1-i- \frac{(+192i+564)(-4+2i)}{1305}\\
 0   & i     & 0   & \vline & 2i-2- \frac{(+192i+564)(-2+6i)}{1305}\\
 0   & 0     & 1   & \vline & \frac{+192i+564}{1305}\\
\end{bmatrix}
\]
we calculate $x_2$ as
\[
x_2= -i\left( \frac{(2i-2)(3^2+6^4)-(+192i+564)(-2+6i)}{3^2+6^4} \right)=\frac{-330-390i}{1305}
\]
and $x_1$ as:
\[
x_1=  \frac{\left[(-1-i)(1305)-(+192i+564)(-4+2i)     \right](3-i)}{1305(3+i)(3-i)}\]\[
=     \frac{\left[(-1-i)(1305)-(+192i+564)(-4+2i)     \right](3-i)}{13050}\]\[
=     \frac{\left[-1305-1305i+768i +2256+384 -1128i   \right](3-i)}{13050}\]\[
=     \frac{\left[-1305-1305i-360i +2640              \right](3-i)}{13050}\]\[
=     \frac{(1335-1665i)(3-i)}{13050}\]\[
=     \frac{2340-6330i}{13050}
%1335*3= 4005
%-1665i*3= 3000 + 1995= -4995i
% -4995i-1335i= -6330i
%4005 -1665= 2340
\]
whith these amazing numbers we can write the solution set in the form:
\[
S=\frac{1}{1305}
\begin{bmatrix}
 \frac{2340-6330i}{10} \\ -330-390i \\ 192i+564
\end{bmatrix}
\]
\section{solve in $\mathbb{R} $ in dependence on $\lambda \in \mathbb{R} $}
\subsection*{2}
\[
\begin{bmatrix}
 \lambda & 1 & 1 & 1 & \vline & 1 \\
 1 & \lambda & 1 & 1 & \vline & 1 \\
 1 & 1 & \lambda & 1 & \vline & 1 \\
 1 & 1 & 1 & \lambda & \vline & 1 \\ 
\end{bmatrix}
\]
we notice that the problem is symmetric, subtracting any two rows yields:

\[
(1-\lambda)x_i= (1-\lambda)x_j
\]
so in the case $\lambda=1$ we have three free variables and the solutionset is:
\[
S=
a\begin{bmatrix}
  1\\0\\0\\-1\\
 \end{bmatrix}
+b\begin{bmatrix}
   0\\1\\0\\-1\\
  \end{bmatrix}
 +c\begin{bmatrix}
    0\\0\\1\\-1\\
   \end{bmatrix}
\quad a,b,c\in \mathbb{R}.
\]
in the case $\lambda \neq 1$ we get $x_1=x_2=x_3=x_4$ which in turn constrains $\lambda$ to:
\[
x = \frac{1}{\lambda+3}
\]
 which gives the solutionset
 \[
 S= \frac{1}{\lambda+3}
 \begin{bmatrix}
  1\\1\\1\\1
 \end{bmatrix}
 \]
\section{check if the following matrices in $\mathbb{R}$ are invertible and compute their inverse if posible }
\subsection{a}
\[
\begin{bmatrix}
 23 & 21 \\
 435 & 45\\
\end{bmatrix}
\]
\[
det 
\begin{bmatrix}
 23 & 21 \\
 435 & 45\\
\end{bmatrix}
= 23\times45-435\times21= 8100
\]
by staring at the 21 it is obvious that the determinant is nonzero, hence we start to compute eigenvalues:
set \[\det \mathbf{A} -\lambda \mathbf{I}\]
\[
(23-\lambda)(45-\lambda)=21\times435
\]
\[
\lambda^2 -69\lambda + 1035 -9135=0
\]
\[
\lambda^2 -69\lambda - 90^2
\]
% 23*45= 1035
% 21*435= 9135
\[
\lambda_{1,2}= \frac{69}{2} \pm \sqrt{(\frac{4761+32400}{4})}= \frac{69}{2} \pm \sqrt{(\frac{4761+32400}{4})}
\]
% 69^2= 4900-140+1= 4761
%32400+4761=37161
\[
\mathbf{A}^{-1}= \frac{1}{det \mathbf{A}} 
\begin{bmatrix}
 45 & -435 \\
 -21 & 23\\
\end{bmatrix}
= \begin{bmatrix}
 \frac{45}{8100}  & -\frac{435}{8100} \\
 -\frac{21}{8100} & \frac{23}{8100}\\
\end{bmatrix}
\]
\subsection*{b}
the inverse matrix of 
$
\begin{bmatrix}
 37
\end{bmatrix}
$
exists and is 
$
\begin{bmatrix}
 \frac{1}{37}
\end{bmatrix}
$
\subsection*{c}
\[
det \begin{bmatrix}
     17 & 24  & 215 \\
     15 & -10 & 119\\
     1  & 17  & 48\\
    \end{bmatrix}
=
(17\times(-10)\times119)+(24\times 119\times1)+(215\times15\times17)-(17\times119\times 17)-(24\times 15\times 48)-(215\times(-10)\times1)
\]
\[
= -20230+2856+ 54825 
\]

\end{document}
